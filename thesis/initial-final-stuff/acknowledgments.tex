%«È di cattivo gusto ringraziare il relatore. Se vi ha aiutato ha fatto solo il suo dovere» Uberto Eco, Come si fa una tesi di laurea

Al termine del mio percorso di studi, vorrei porgere i miei più profondi ringraziamenti
a tutte le persone che hanno contribuito, in misura più o meno significativa, alla mia
crescita professionale e personale.
Inizio ringraziando la mia famiglia, il cui supporto economico e morale non è mai mancato.
In particolare, sono estremamente grato a mia madre per avermi dato la possibilità
di focalizzarmi sull’università, investendo parte del suo tempo per offrirmi
le migliori condizioni di studio possibili.
Ringrazio mio padre, per gli insegnamenti che mi ha trasmesso e per come, mediante l’esempio,
mi ha mostrato ad affrontare le sfide con passione e tenacia.
Un pensiero particolare va a Ilaria, mia sorella, una donna con dei sani valori,
testarda e irascibile come poche, ed in grado di raggiungere grandi traguardi.
Mi auguro che possa ricevere dalla vita i premi del duro lavoro che ha compiuto e che sarà chiamata a svolgere.
Ringrazio Antonio, mio compagno d’avventure in questi anni, per tutte le
esperienze vissute insieme e, soprattutto, per non essere mai mancato nei momenti di difficoltà.
Un grazie va poi ad Andrea, che è stato una presenza costante nella mia permanenza milanese,
un uomo sempre disponibile per un consiglio e sempre pronto a tendermi la mano.
Ringrazio Iris per la pazienza dimostrata e le sono grato per l’energia che è
riuscita a trasmettermi.
Le auguro il meglio per la sua vita e sono sicuro che sarà capace di raggiungere
tutti i traguardi che si è prefissata.
Non posso non considerare l’aiuto indispensabile che Veronica ha voluto darmi in questi mesi.
Nonostante la mole di lavoro a cui era sottoposta, ha saputo trovare tempo da dedicarmi.
Vorrei quindi ringraziarla profondamente.
Porgo i miei sentiti ringraziamenti al professor Lovera, per avermi accompagnato
durante la stesura della tesi e per la pazienza dimostrata in alcune occasioni.
Lo ringrazio per la stima e per la fiducia che ha riposto in me.
Un ringraziamento va a Mattia, che mi ha fornito un supporto indispensabile
nello svolgimento pratico della tesi, offrendomi le sue competenze ingegneristiche
e la sua esperienza con i droni.
Infine, vorrei ringraziare Pietro per tutto ciò che ha fatto per me.
Non si è rivelato solamente un supervisore, ma ha saputo essere un punto saldo
a cui sorreggersi nei momenti difficili. Lo ringrazio di cuore
per il suo altruismo disinteressato e non dimenticherò mai quello che ha fatto per me. 
