%Il testo delle tesi redatte in lingua straniera dovrà essere introdotto da un ampio estratto in lingua italiana, che andrà collocato dopo l’abstract.

La tesi tratta di volo sincronizzato di una formazione di multirotori che eseguono una missione,
caratterizzata da una traiettoria per ogni drone che compone la formazione.
La formazione deve essere in grado di reagire a eventi inaspettati, che posssono
compromettere l'esito della missione stessa.
Per raggiungere tale scopo, i droni devono scambiare informazioni, attraverso
una rete di supporto.

L'obbiettivo della tesi è di presentare l'algoritmo utilizzato per il volo sincronizzato,
chiamato “Algoritmo di Consenso” e implementarlo sia in un ambiente simulato,
che in un sistema reale, composto da macchine eterogenee. Si vuole verificare che
i risultati teorici possano essere applicati in un sistema reale distribuito, dotato di
una rete di comunicazione dalle prestazioni non ideali.

Nella prima parte della tesi, verrà esposto l'algoritmo usato durante il lavoro sperimentale,
mentre nel capitolo seguente saranno presentati l'hardware e il software.
Negli ultimi capitoli saranno evidenziati nel dettaglio la struttura del software
e gli esperimenti condotti.
In praticolare, saranno mostrate alcune simulationi, in modo da confermare la bontà
dell'“Algoritmo di Consenso”.
Infine, nell'ultimo capitolo, sarà proposta una comparazione tra i risultati
delle simulazioni e quelli ottenuti mediante l'implementazione in un sistema reale.
