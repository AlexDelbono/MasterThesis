%The introduction must be atomic, it does not have to include subsections nor paragraphs.

%The title, the abstract and the introduction must appear as chinese boxes: if read in this order, they must progressively show the informations about the topic in order to catch the attention of the reader.

%The introduction must be divided in three main logic blocks as well:
%General overview
%A brief description of the work
%Structure of the thesis

These years are characterized by the development of the robotics.
The growing applications of mobile robots and drones in many fields
has brought important increasing in the number of robotic vehicles.
Many commercial solutions are available, but, in many cases, they offer
a product which needs an expert pilot. For this reason and due to the high prices,
the diffusion of mobile robots is limited. In the latest years, these problems were
mitigated and we now find lower prices and easier-to-use products. This will bring a
growing non professional diffusion of robotic vehicles which, hopefully, will populate
our houses and help us in our life.
The most important application of UAVs are the ones related to exploration and data
collection. Indeed, drones are fundamental in inspection of unknown areas, such as
forests or unreachable terrains.
Also the monitoring of industrial artifacts constitutes a valuable application of UVAs.
For instance, solar panels, wind turbines, high-voltage cables or tubes can be
supervised using aerial vehicles.
The multimedia production is the field in which the drones are most present. Indeed,
most of the machines are equipped with high-definition cameras in order to provide
professional videos.

The power consumptions of the machines are significantly reduced in the last years
and we are now able to build smaller and lighter vehicles.
In particular, in the aerial field, there are commercial products which weight
about 200g with a flight time of 20 minutes.
These improvements allow us to develop more advanced features, such as
trajectory planning, obstacles avoidance, formation flight.
The development of autonomous UAVs is fundamental in application such as humanitarian
response around the world. When a natural calamity happens, the UAVs can play a
crucial role, delivering essential goods or finding missing people. Moreover, the
operations can be done without risking the safety or the life of the rescuers, because
they can act remotely or plan an autonomous mission.

The trajectory planning is essential for the development of autonomous vehicles,
because the generation of a feasible trajectory for a mission is a requirement,
otherwise no mission can be done.
The trajectory generation is an hard problem, because of its computational complexity.
Indeed, many suboptimal algorithm have been developed in order to reduce the
complexity.
There are different classes of planning algorithm and they can be summarized as follows:
\begin{itemize}
  \item Artificial Potential Field: these algorithms assign a value of the potential
  for every point in the map. The goal has the lowest (highest) value and the obstacles
  have high (low) values. Then, the robot tries to descend (climb) the potential,
  in order to reach the goal.

  \item Sampling Based Planning: random samples are used to find a path from the
  starting point to the goal. Advanced random samples techniques can be used in order
  to reduce the complexity of the algorithm.

  \item Grid Based Planning: these kind of algorithm overlay a grid on the map, so
  every configuration corresponds with a grid pixel.
  The robot can move from one grid pixel to any adjacent grid pixels as long
  as that grid pixel does not contain an obstacle.

  \item Reward-Based Planning: the robot can apply different actions in every state
  of the world. The outcome of an action can be not deterministic and after every
  action, the robot gains a reward. The objective of the robot is to maximize the
  sum of the rewards.
\end{itemize}

Another important aspect is the obstacle avoidance. The trajectory usually is planned
offline, when the mission is not started yet. So, the trajectory does not take into
account the fact that an unplanned obstacle might interfere with the mission. In this
case another algorithm is implemented in order to run online. This kind of algorithms
are called obstacle avoidance algorithms and they replan locally the trajectory, in
order to get rid of unforeseen events. These algorithms need the drone to be equipped
with proximity sensors, which provide information about the surrounding environment.
Different methodologies have been developed for obstacles avoidance. The main ones are
summarized below without details.

\begin{itemize}
  \item Artificial Potential Field: as before, the idea of this approach is that
  obstacles exert a virtual repulsive force, given by the potential field, to push
  away the robot from them and the goal position generates a virtual attractive
  force to guide the robot to it.

  \item Virtual Force Field method (VFF): is the combination of the Artificial Potential
  Field with the concept of Probabilistic Occupancy Grid maps.

  \item Fuzzy Controller for Obstacle Avoidance: as the name says, a fuzzy controller
  is used to derive the variables for the vehicle’s orientation and acceleration control,
  depending on the current perception of the robot’s sensors

  \item Vector Field Histogram method (VFH): improved version of the VFF method.

  \item VFH+, VFH$^*$ method: improved versions of the VFH method, but more computational costly.
  The most recent one is the VFH method, which uses an A$^*$ search.

  \item Traversability Field Histogram (TFH): The local path planner bases on
  the VFH concept, but is extended by the information provided within the Traversability Map.

  \item Dynamic Window Approach: the algorithm takes into account the dynamic and kinematic constraints
  of the robot and it is similar to the VFH+ algorithm. The method is called
  dynamic window approach and considers only admissible velocities which can be
  reached within the next time interval and which allow the robot to stop safely.

  \item Curvature Velocity Space method: This method chooses a point in the linear-angular
  velocity space which satisfies some constraints and maximizes an objective function.
  This objective function tries to move the robot close to the commanded direction at
  the highest feasible speed, while travelling the trajectory with the largest clearance from obstacles.

  \item Beam Curvature method: improvement for the curvature velocity space method.

  \item Nearness diagram Navigation: it is similar to the VFH method, but uses
  a polar histogram to derive actions to be taken for the robot.
\end{itemize}

When multiple machines have to fly in a formation, a synchronization mechanism is
needed. Indeed, if one of the drones has to deviate from the planned trajectory
(for instance, because of an unplanned obstacle), the formation must be preserved.
To reach the synchronization, the vehicles must communicate with each others and
exchange information.
Notions from graph theory are needed to analyze the behaviour of the system
and demonstrate the convergence properties.

This work is focused on the presentation of a distributed synchronization algorithm,
which we call “Consensus Algorithm”. Every drones must exchange its mission progression
with its neighbors and must adjust its mission progression based on the information
received from the other vehicles.
The algorithm is a distributed algorithm and it is executed on every vehicles.


In the first chapter, we will show the theoretical aspects of the “Consensus Algorithm” and then,
in the second chapter, we will provide a general overview of the hardware and software used to deploy the algorithm.
In the third chapter we will show the software implementation of the main part of the
system, providing classes and snippets of code.
At the end, we will explain in the fourth and fifth chapters the simulated and experimental results of
our work. We will apply the algorithm to a real system and we will present our results and its
performances.
