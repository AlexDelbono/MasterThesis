%The conclusions must recall the field of work, the purpose of the thesis, what has been done and an evaluation of the obtained results.
%Furthermore, the conclusion must also emphasize the future prospects and must show how to move forward in the study area.

The results obtained in our experiments can be applied in many fields in which
a formation of UAVs is operating.
In this thesis, we have shown how the consensus algorithm is deployed on a real
system and what performances can be achieved.

The implementation of the consensus algorithm allows us to plan and fulfil a cooperative
mission in which the synchronization is one of the most important aspects.
The system is robust to network delays, loss of packets and to unexpected failures
of the vehicles involved.
Moreover, if equipped with an obstacle avoidance algorithm, the formation
overcomes the presence of unexpected obstacles during the execution of the mission.

Further studies can be carried out in order to integrate recovery procedures in case
of network failures or loss of a machine.
It is possible to develop obstacle avoidance algorithms or any kind of online procedures
acting directly on the free parameters of the algorithm presented.
In particular, it is possible to change the values of $\ddot{\gamma_d}$ and $\dot{\gamma_d}$,
in order to modify the velocity of the execution of the mission. These two values
can be also used to build more complex functionalities on top of the consensus algorithm.
