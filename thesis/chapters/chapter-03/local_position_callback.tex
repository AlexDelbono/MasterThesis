\section{Local position callback\label{sec:local_position_callback}}

In the position callback function we apply the consensus law. First of all, we
store the actual position of the drone in an object of a custom class,
which signature is presented in figure \ref{fig:drone_pose}.
\begin{figure}
\centering
  \lstinputlisting[language=C++]{chapters/chapter-03/code/drone_pose.hpp}
\caption{Class used to manage the position and velocity of the drones}
\label{fig:drone_pose}
\end{figure}
In this class, we also include, besides x, y and z, the yaw of our vehicle. Indeed,
all the operations defined over the class consider also the orientation.

Now, we compute the synchronization term which corresponds to the sum of the
difference between the current $\gamma_i$ and all the $\gamma_j$ of the neighbors.
In order to do this, we iterate over a container, which stores the values, and
we incrementally form the synchronization term.

The next step is to form the $\overline{\alpha}_i$ term.
First of all, we need the next set point, which must be obtained evaluating
the trajectory with the actual value of $\gamma_i$.
Our trajectory is represented by a class, and its signature is shown in the figure
\ref{fig:loc_pos_callback_trajectory}.
\begin{figure}
\centering
  \lstinputlisting[language=C++]{chapters/chapter-03/code/trajectory.hpp}
\caption{Class used to manage a generic trajectory}
\label{fig:loc_pos_callback_trajectory}
\end{figure}
Now, we simply compute the $position\_error$ as $set\_point - position$.
We also need the desired velocity, which can be obtained using the suitable function
of the trajectory class.
At this point, we have all the terms needed to compute $\overline{\alpha}_i$
as defined in \ref{eq:error_term}.

Since we have all the elements, it is time to apply the consensus law and find
$\ddot{\gamma}_i$. We simply need to have the coefficients $a$ and $b$
and the references $\ddot{\gamma}_d$ and $\dot{\gamma}_d$.

One of the last step needed is the update of $\dot{\gamma}_i$ using $\ddot{\gamma}_i$
and $\gamma_i$ using $\dot{\gamma}_i$. We compute the interval of time, $dt$, between the
last update and the current update and we do the math as:
\begin{lstlisting}
    dgamma += ddgamma * dt;
    gamma += dgamma * dt;
\end{lstlisting}

At the end, we need to publish the value of $\gamma_i$ to the right topic only
if we are operating in consensus mode, otherwise we ignore it. An operation which is
always needed is the publication of the setpoint message to the autopilot of
the UAV, in order to allow it to follow the trajectory and reach its final destination.
