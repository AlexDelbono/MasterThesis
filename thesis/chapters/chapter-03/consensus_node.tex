In this chapter, we will examine the structure of the consensus node. We will see
its main components and we will show some snippets of code
in order to make clearer which parts are involved.

As shown in the general architecture (chapter \ref{chap:system_architecture}),
the consensus node is developed as a ROS node, which subscribes and publishes messages
to different topics.
Moreover, the node offers some ROS services used to start and to stop the trajectory
following algorithm or the consensus algorithm.

The structure of the node takes into account the main architectural patterns used
in the software development field and it was designed to allow the maximum degree
of usability and customization. However, since it has to be executed on
machines with limited amount of resources, one of the most important metric taken
into account is the efficiency of the code.

The node functionalities are enclosed into a C++ class which initializes
all the ROS elements and prepares the node to receive the start and stop commands.
The initialization is done by the class constructor when the object is created.
First, in order to apply the consensus dynamic equation, we need
the current position of the UAV. The Px4 board already publishes the estimated
local position on a topic.
Therefore, all we need to do is subscribe to that topic to retrieve the messages
with the required information.
Second, we want to publish the consensus variable of the drone in the topic used
by all the other UAVs, because, having obtained the others' consensus variables
from the same topic, we are able to compute the proportional consensus error.
Third, since we want to compute the position error and weight
it for the target velocity, we need the next set point and the next desired velocity profile.
Finally, we compute the acceleration of the consensus parameter using the consensus equation (\ref{eq:cons_law})
and we publish the next set point. We will see the details through the code.
All these elements can be summarized and shown in Figure \ref{fig:node_in_out}.

\begin{figure}
\centering
\includegraphics[width=0.8\textwidth]{chapters/chapter-03/figures/consensus_node_structure.pdf}
\caption{Input and output of the node}
\label{fig:node_in_out}
\end{figure}

The subscription of a ROS topic works through a callback function, which accepts as
parameter the pointer to the new message. Since in our case we have multiple subscriptions
and we must advertise the start and stop services, we need to implement a multithreading
architecture which takes care of the concurrent accesses to the state of our object.
Three threads have been used and their functions are listed below:
\begin{itemize}
  \item Start and stop services
  \item Consensus variable callback
  \item Local position callback
\end{itemize}

%% Sections of the chapter
\section{Start and stop services\label{sec:start_stop_services}}


\section{Consensus variable callback\label{sec:consensus_variable_callback}}

The thread responsible for collecting the consensus variables of all the other UAVs,
is managed by ROS and it executes a callback function when a new message is published
on a specific topic.
This topic is used by all the drones to publish their consensus variable, $\gamma_i$,
and it accepts a custom message which contains only a string with the name of the
owner of the variable and the value itself. The message has also a header, which contains
general information such as timestamp or message ID.
The structure of the message can be seen in the Figure \ref{fig:custom_message}.

\begin{figure}
\centering
  \lstinputlisting{chapters/chapter-03/code/GammaMsg.msg}
\caption{Custom message structure}
\label{fig:custom_message}
\end{figure}

The callback function receives the information from the topic and updates a local
view of the variables of the neighbors. This information has a timeout validity,
because we do not want to consider values which are too old. Indeed, if we considered too old
values and a problem in the network caused a loss of packets, our drone might
think that the other drones have significantly different values of the consensus variables
and might therefore wait for them. This is why, it is better to discard these values and
remove the neighbors after a timeout.

In order to store this information, we use a thread safe support class, which
provides a procedure to check if the variable is expired or not.
The signatures of the methods of the class are presented in the Figure
\ref{fig:consensuss_variable_class}.
We use a container to store the values of the neighbors and we always check if the
value has expired or not before using it.

\begin{figure}
\centering
  \lstinputlisting[language=C++]{chapters/chapter-03/code/GammaParameter.cpp}
\caption{Consensus variable class}
\label{fig:consensuss_variable_class}
\end{figure}

These variables are used in the consensus law (\ref{eq:cons_law})
as $\gamma_j$ of the neighbors and are used to compute the proportional error.
We can see that the expiration interval of the values can model the
fact that the network topology can change.
Indeed, if a link between two drones vanished because, for instance, they were
too far from each other, after a timeout (which it is equal to the expiration time),
the neighbor would be removed from the container and the drone would not take into
account the old neighbor.
The timeout can even ignore a failure of a drone: if a machine had a critical
problem and did not send its consensus variable, the other drones would remove it from
their neighbors and continue their mission without problems.


\section{Local position callback\label{sec:local_position_callback}}

In the position callback function we apply the consensus law. First of all, we
store the actual position of the drone in an object of a custom class,
which signature is presented in figure \ref{fig:drone_pose}.
\begin{figure}[ht]
\centering
  \lstinputlisting[language=C++]{chapters/chapter-03/code/drone_pose.hpp}
\caption{Class used to manage the position and velocity of the drones}
\label{fig:drone_pose}
\end{figure}
In this class, we also include, besides x, y and z, the yaw of our vehicle. Indeed,
all the operations defined over the class consider also the orientation.

Now, we compute the synchronization term which corresponds to the sum of the
difference between the current $\gamma_i$ and all the $\gamma_j$ of the neighbors. %%TODO ref
In order to do this, we iterate over a container, which stores the values, and
we incrementally form the synchronization term.

The next step is to form the $\overline{\alpha}_i$ term. %%TODO ref
First of all, we need the next set point, which must be obtained evaluating
the trajectory with the actual value of $\gamma_i$.
Our trajectory is represented by a class, and its signature is shown in the figure
\ref{fig:trajectory}.
\begin{figure}[ht]
\centering
  \lstinputlisting[language=C++]{chapters/chapter-03/code/trajectory.hpp}
\caption{Class used to manage a generic trajectory}
\label{fig:trajectory}
\end{figure}
Now, we simply compute the $position\_error$ as $set\_point - position$.
We also need the desired velocity, which can be obtained using the suitable function
of the trajectory class.
At this point, we have all the terms needed to compute $\overline{\alpha}_i$
as defined in -TODO-. %%TODO reference

Since we have all the elements, it is time to apply the consensus law and find
$\ddot{\gamma}_i$. We simply need to have the coefficients $a$ and $b$, as shown in
-TODO-, and the references $\ddot{\gamma}_d$ and $\dot{\gamma}_d$.

One of the last step needed is the update of $\dot{\gamma}_i$ using $\ddot{\gamma}_i$
and $\gamma_i$ using $\dot{\gamma}_i$. We compute the interval of time, $dt$, between the
last update and the current update and we do the math as:
\begin{lstlisting}
    dgamma += ddgamma * dt;
    gamma += dgamma * dt;
\end{lstlisting}

At the end, we need to publish the value of $\gamma_i$ to the right topic only
if we are operating in consensus mode, otherwise we ignore it. An operation which is
always needed is the publication of the set point message to the autopilot of
the UAV, in order to allow it to follow the trajectory and reach its final destination.

