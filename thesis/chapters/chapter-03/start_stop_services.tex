\section{Start and stop services\label{sec:start_stop_services}}

First of all, the node can work in two different modes:
\begin{itemize}
  \item trajectory-only
  \item consensus
\end{itemize}

In the trajectory-only mode, the node computes the next set point and sends it to
the UAV, without taking care if there are others UAVs in the mission. Its only objective
is to follow the trajectory and reach its final position trying to respect the time
constraints imposed by the trajectory.
Instead, in the consensus mode, the node does the same computation as before, but
it also publishes its consensus variable and reads all the other ones.
It then considers this information and adjusts its consensus variable.
The consensus mode includes the trajectory-only, and can be started even if the
trajectory-only is already started, while the opposite is not true. When you stop
the consensus mode, also the trajectory-only is stopped.
When the mission is accomplished, the current active mode is stopped automatically,
so that you can freely restart one of the two modes without the need of stopping
the previous one.

The services are implemented using the ROS Service class, which manages the whole
infrastructure needed for calling the service. The call of a service is
synchronous and the caller is blocked until the service function is terminated.
In this case, we offer two services: one for starting and stoping the trajectory-only
mode and the other for the consensus mode.
It is possible to customize the service call in order to pass different number
and types of arguments to the service function and the response can also be
defined.
For the two services we have defined the same parameters that are shown in the figure
\ref{fig:custom_service}.

\begin{figure}
\centering
  \lstinputlisting{chapters/chapter-03/code/FlyartCommandBool.srv}
\caption{Custom service structure}
\label{fig:custom_service}
\end{figure}

The message is composed by two parts: the request and the response.
In the request we need a boolean field in order to know if
we want to start or stop the algorithm. The response consists in a boolean variable,
which represents the success of the operation, and an exit code, which identifies
the eventual problems occurred. The constants for the exit codes are directly specified
in the definition of the service.

The trajectory-only service starts or stops the thread which, taken a local position,
computes the next setpoint; while the consensus one starts or stops the same
thread as before and the one which retrieves the consensus variables from the other
quadrotors.
