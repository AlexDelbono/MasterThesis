\section{Consensus variable callback\label{sec:consensus_variable_callback}}

The thread responsible for collecting the consensus variables of all the other UAVs,
is managed by ROS and it executes a callback function when a new message is published
on a specific topic.
This topic is used by all the drones to publish their consensus variable, $\gamma_i$,
and it accepts a custom message which contains only a string with the name of the
owner of the variable and the value itself. The message has also a header, which contains
general information such as timestamp or message id.
The structure of the message can be seen in the figure \ref{fig:custom_message}.

\begin{figure}[h]
\centering
  \lstinputlisting{chapters/chapter-03/code/GammaMsg.msg}
\caption{Custom message structure}
\label{fig:custom_message}
\end{figure}

The callback function receives the information from the topic and updates a local
view of the variables of the neighbors. This information has a timeout validity,
because we do not want to consider too old values. Indeed, if we consider too old
values and a problem in the network causes a loss of packets, our drone might
think which the other drones have a significantly different values of the consensus variables
and so, wait for them. On the contrary, it is better to discard the values and
remove the neighbors after a timeout.

In order to store the information, we use a support class, thread safe, which
provides a procedure to check if the variable is expired or not.
The signatures of the methods of the class are presented in the figure
\ref{fig:consensuss_variable_class}.
We use a container to store the values of the neighbors and we always check if the
value is expired or not before using it.

\begin{figure}[ht]
\centering
  \lstinputlisting[language=C++]{chapters/chapter-03/code/GammaParameter.cpp}
\caption{Consensus variable class}
\label{fig:consensuss_variable_class}
\end{figure}

These variables are used in the consensus law %%TODO reference
as $\gamma_j$ of the neighbors and are used to compute the proportional error.
We can see that the the expiration interval of the values can model the
fact that the network topology can change.
Indeed, if a link between two drones vanished because, for instance, they are
too far from each other, after a timeout (which it is equal to the expiration time),
the neighbor will be removed from the container and the drone will not take into
account the old neighbor.
Even a failure of a drone is ignored using the timeout: if a machine has a critical
problem and does not send its consensus variable, the other drones remove it from
their neighbors and, therefore, they can continue their mission without problems.
