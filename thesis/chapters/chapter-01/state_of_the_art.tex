%%Introduction of the chapter
The literature about the "Consensus" has grown significantly in the latest years
because of the increasing presence of autonomous vehicles. This, in accordance with
the new improvements in robotics, has brought to growing interest in consensus
between multiple agents which have to accomplish a mission in cooperative or adversarial
scenarios.

The "Consensus" theory has its roots in Graph Theory and Automatics and it can
be used to coordinate a mission in order to achieve the synchronization between
the vehicles even during unforeseen events which force one or more
components of the mission to change the planned trajectory or task. In this case,
the other components identify this variation and they act to preserve the
synchronization.

The "Consensus" algorithm is a distributed algorithm which can sometimes be simulated
in a centralized fashion because of the reduced computational power of the machines
involved in the mission, which are equipped with low power hardware to account for
the crucial issue of power consumption.
In this scenario, a centralized server simulates the algorithm and communicates with
all the machines.

The most common "Consensus" application is a spatial and timing consensus: in this
implementation, each vehicle of the formation has to travel along a specified trajectory and the
completion of the mission occurs when all the agents reach the final positions of their
spatial paths. The algorithm has to guarantee that the difference between
the ending time at which all the vehicles finish their tasks is minimized and
asymptotically goes to zero, when the execution time goes to infinity and no other
unforeseen events happens.
We also consider formations of Unmanned Aerial Vehicles, but the key concepts can
be freely applied to other categories of robots when they are able to follow a trajectory.
This study is focused on this kind of application, and further
simulation results and experimental achievements are presented in chapters
\ref{chap:simulation_results} and \ref{chap:experimental_results}.

In the next sections we refer to the main work of Venanzio Cichella \cite{cichellaMain}
in the field of UAV consensus, in order to provide an homogeneous state of the art
about the "Consensus".
The paper considers all the details needed to build a "Consensus" system.
The main components of this kind of system are listed and explained below:
\begin{itemize}
  \item Parametrized trajectory
  \item Virtual time
  \item Consensus law
  \item Network topology
\end{itemize}



%% Sections of the chapter
\section{Parametrized trajectory\label{sec:parametrized_trajectory}}
The trajectory is a spatial path with an associated timing law and it is used
to identify the position of the center of mass of our agents at a given time.
First of all, the trajectory used during our experiments are polynomials curves.
In particular, we use Bézier curves, because of the existence of many
computationally efficient algorithms designed for this kind of curves,
such as algorithms to efficiently compute the minimum distance between two Bezier curves
and the existence of a closed-form solution for the arc lengths of the paths.

First of all, let $I = \{N, E, D\}$ denote a right-handed inertial frame with $N$, $E$ and $D$
unit vectors along north, east and down respectively.
The vector $p = (x, y, z) \in I$ denotes the position of the center of mass of the vehicle.
Let $B = \{X_B, Y_B, Z_B\}$ be right-handed body fixed frame centered in the center of mass of the vehicle.
The orientation of the rigid body is given by rotation matrix $R(\phi, \theta, \psi)$ where
$\phi$, $\theta$ and $\psi$ are the roll, pitch and yaw Euler angles respectively.
We can state the following definition of a generic trajectory $ \bm{p}_{d,i}(t_d) $
for $ N $ vehicles:

\begin{equation}  \label{eq:traj_def}
  \bm{p}_{d,i}:[0,t^f_{d,i}] \rightarrow \mathbb{R}^4, \quad i = 1,2,\dots,N
\end{equation}
where $ t_d \in [0, T_d] $, with $T_d := max \{ t^f_{d,1}, \dots , t^f_{d,N} \} $,
is the time variable of the trajectory, while $ t^f_{d,i} \in \mathbb{R}^+ $
are the individual final mission times of the vehicles obtained during the planning
phase. Usually all these final times are equal and therefore
$ t^f_{d,1} = \dots = t^f_{d,N} = T_d$, but we introduced the notation
for the sake of generality.
Obviously, the trajectories need to be collision-free and must comply with spatial
and temporal constraints due to the dimensions of the vehicles and their maximum
velocities and accelerations.
The range of the function is $\mathbb{R}^4$ because, in the case of UAVs, we want
to control $x$, $y$, $z$, $\psi$.
The trajectory can account also for $roll$ and $pitch$, and therefore, the function might take
images in $\mathbb{R}^m$, where $m$ is the number of dimensions considered.
In the following sections, we will refer only to the position and $\psi$, but a more general
theory can also be developed.

We now parametrize the trajectory using a dimensionless variable $ \zeta_i \in [0,1]$,
related to the time $t_d$. In this way we can specify a function $ \theta( \cdot )$,
which represents the timing law associated with the spatial path $\bm{p}_{d,i}(\zeta_i)$.
We can specify this timing law using the dynamic relation:

\begin{equation}  \label{eq:tim_law_def}
  \theta( t_d ) = \frac{d \zeta_i}{dt_d}
\end{equation}

where $ \theta( t_d ) $ is a smooth and positive (the parameter increases when
the time increases) function.

As it allows a one-to-one correspondence between the time
variable $t_d$ and the parameter $ \zeta_i $, an analytical expression
for the function $ \zeta_i (t_d) $ is desirable.
Using the timing law defined in \eqref{eq:tim_law_def}, the map $ \zeta_i ( t_d) $
is given by the integral:

\begin{equation}  \label{eq:zeta_law_def}
  \zeta( t_d ) = \int^{t_d}_0 \theta_i(\tau) d \tau
\end{equation}

Usually, all these functions are defined as polynomials in order to make the evaluation process
quicker and easier, since multiplication and addition are the basic
operations in a digital processor. However, it is not mandatory to use them and a
generic shape for the functions can be designed.

For the spatial path we use quintic Bezier curves for the $x$, $y$ and $z$ coordinates,
while we use a third degree polynomial for the $\psi$ of the drone since we need less
elaborate shapes.
Since we are using drones with a decoupled control of $x$, $y$, $z$ and $\psi$, the
polynomials are completely independent.

The timing law associated to these curves is a Bezier third order polynomial.
So, in our case, we have five polynomials which have to be evaluated in real time
in order to provide the set points. The frequency at which we evaluate the trajectory
is $5 Hz$.

The trajectory which describes the mission of a single drone is given by:

\[
  \begin{cases}
    x_i(\zeta_i) = \sum_{j=0}^{5}{\bar{x}_{i,j} b_j^5(\zeta_i)}\\
    y_i(\zeta_i) = \sum_{j=0}^{5}{\bar{y}_{i,j} b_j^5(\zeta_i)}\\
    z_i(\zeta_i) = \sum_{j=0}^{5}{\bar{z}_{i,j} b_j^5(\zeta_i)}\\
    \psi_i(\zeta_i) = \sum_{j=0}^{3}{\bar{\psi}_{i,j} b_j^3(\zeta_i)}\\
    \zeta_i(t_d) = \sum_{j=0}^{3}{\bar{\zeta}_{i,j} b_j^3(t_d)}\\
  \end{cases}
\]

where $\bar{x}_{i,j}$, $\bar{y}_{i,j}$, $\bar{z}_{i,j}$, $\bar{\psi}_{i,j}$ and $\bar{\zeta}$ $\in \mathbb{R}$
are the desired control points of the spatial path and $b_n^m$ are the (up to degree $m$)
Bernstein basis polynomials.
We do not enter in details about the Bezier curves and Bernstein basis polynomials,
because it is not the objective of this work.

We have defined all the elements of a trajectory and we do not go into detail
about the trajectory generation phase.
Further information about boundary conditions and flyable trajectory, which satisfy
the dynamic constraints of the vehicles, can be found in \cite{cichellaMain} and
is extensively analyzed in \cite{trajGeneration1}, \cite{trajGeneration2},
\cite{trajGeneration3}.


\section{Virtual time\label{sec:virtual_time}}


\section{Consensus law\label{sec:consensus_law}}

Now, we formally state the path following problem. We define as $\bm{p}_i(t) \in \mathbb{R}^4$
the position of the center of mass of the $i^{th}$ agent and its $yaw$ and since $\bm{p}_{c,i}(t)$
describes the commanded pose to be followed by the agent at time $t$, the errors
are defined as:

\begin{equation}  \label{eq:pos_vel_acc_err}
  \begin{aligned}
  \bm{e}_{p,i}(t) = \ & \bm{p}_{c,i}(t) - \bm{p}_i(t) \in  \mathbb{R}^4\\
  \bm{e}_{v,i}(t) = \ & \bm{v}_{c,i}(t) - \dot{\bm{p}}_i(t) \in  \mathbb{R}^4
  \end{aligned}
\end{equation}

Then, the objective reduces to that of regulating the error defined in \eqref{eq:pos_vel_acc_err}
to a neighbourhood of zero.
This task is solved with an autopilot capable of following the set points computed
from the desired trajectory at specified instances of time.

The virtual time is the parameter used to reach consensus between multiple vehicles.
In fact, since the the trajectories are parametrized by $\gamma_i$, the agents are
synchronized at time $t$ when:

\begin{equation} \label{eq:diff_gamma}
  \gamma_i(t) - \gamma_j(t) = 0 \quad for \; all \quad i,j \in {1 , \dots , N}, \quad i \neq j
\end{equation}

We can also control the rate of progression of the mission using a parameter
$\dot{\gamma}_d \in \mathbb{R}$, which represents the velocity of the virtual time
with respect to the real time. All the agents share this variable and they proceed
at the same rate of progression if:

\begin{equation} \label{eq:diff_dot_gamma}
  \dot{\gamma}_i(t) - \dot{\gamma}_d(t) = 0 \quad for \; all \quad i \in {1 , \dots , N}
\end{equation}

Adjusting $\dot{\gamma}_d$, we can decide the speed of the mission: for instance,
if we set $\dot{\gamma}_d = 1$ and \eqref{eq:diff_gamma} and \eqref{eq:diff_dot_gamma}
are satisfied for all the vehicles, then the mission is executed at the speed
originally planned in the trajectory generation phase.
If instead we use $\dot{\gamma}_d > 1$ or $\dot{\gamma}_d < 1$ we carry out the
mission faster or slower.
This term can be changed in real time in order to avoid moving objects or
unplanned obstacles, which make it necessary to change one of the path of the agents.
For the purpose of “Consensus”, the parameter is only a reference command,
rather than a control input.

Now, we introduce the coordination control law which regulates the evolution of
$\ddot{\gamma}_i(t)$ during the time and determines $\gamma_i(t)$:

\begin{equation} \label{eq:cons_law}
  \begin{aligned}
    \ddot{\gamma}_i(t) = & \; \ddot{\gamma}_d(t) - b (\dot{\gamma}_i(t) - \dot{\gamma}_d(t)) - a \sum_{j \in \aleph_i} (\gamma_i(t) - \gamma_j(t)) - \overline{\alpha}_i (\bm{e}_{p,i}(t)) \\
    \dot{\gamma}_i(0) = & \; \dot{\gamma}_d(0) = 1 \\
    \gamma_i(0) = & \; \gamma_d(0) = 0
  \end{aligned}
\end{equation}
where $a$ and $b$ are positive coordination control gains, while $\overline{\alpha}_i (\bm{e}_{p,i}(t))$
is defined as:

\begin{equation} \label{eq:error_term}
  \overline{\alpha}_i (\bm{e}_{p,i}(t)) = \frac{\bm{v}_{c,i}^T(t) \bm{e}_{p,i}(t)}{||\bm{v}_{c,i}(t)|| + \epsilon}
\end{equation}

with $\epsilon$ being a positive design parameter, $\bm{e}_{p,i}$ the position
error vector defined in \eqref{eq:pos_vel_acc_err} and $\aleph_i$ the set of the
neighbors which can communicate with the $i^{th}$ vehicle (we will see details later).
In the equation \eqref{eq:cons_law} we have four terms. The feedforward term,
$\ddot{\gamma}_d$, allows the virtual target to follow the acceleration profile of
$\gamma_d$.
The second term, $- b (\dot{\gamma}_i(t) - \dot{\gamma}_d(t))$, reduces the error
between the speed profile imposed by $\dot{\gamma}_d(t)$ and $\dot{\gamma}_i(t)$,
which corresponds to the control objective given in \eqref{eq:diff_dot_gamma}.
In particular, if $\dot{\gamma}_d(t)$ is one, then the virtual target converges
to the desired speed profile chosen in the trajectory generation phase.
The third term, $- a \sum_{j \in \aleph_i} (\gamma_i(t) - \gamma_j(t))$, ensures that
all the vehicles are coordinated with their neighbors as specified in \eqref{eq:diff_gamma}.
Finally, the fourth term, $- \overline{\alpha}_i (\bm{e}_{p,i}(t))$, is a correction term
used to take into account for the path following errors of the agent. Indeed, if the
vehicle is behind its target, the term is not zero and the target slows down in order
to wait for the real vehicle.

With this control law, we want our vehicles to be synchronized and to proceed at
a desired rate of progression, in order to accomplish the mission even when
some unforeseen disturbances occur during the execution phase.


\section{Network topology\label{sec:network_topology}}

To achieve time-coordination objective, agents must exchange information
over a supporting communication network.
To analyze the information flow, we need to consider some tools from algebraic graph
theory, whose key concepts can be found in \cite{graphBook}.

We assume that a vehicle $i$ exchanges information with only a subset of all
vehicles, denoted as $\aleph_i(t)$.
We assume that arcs of the network are bidirectional and that there are no
network delays.
The information exchanged is composed by the virtual time of the agents,
$\gamma_i(t)$.

The topology of the graph $\Gamma(t)$ that represents the communication network
must comply with the following constraint in order to guarantee the convergence
of the "Consensus" algorithm:

\begin{equation}  \label{eq:graph_cond}
  \frac{1}{NT} \int_t^{t+T} Q L (\tau) Q^T d \tau \ge \mu I_{N-1}, \quad for \; all \quad t \ge 0
\end{equation}
where $L(t) \in \mathbb{R}^{N \times N}$ is the Laplacian of the graph $\Gamma(t)$
and $Q \in \mathbb{R}^{(N-1) \times N}$ is a matrix such that $Q 1_N = 0$ and
$QQ^T = I_{N-1}$, with $1_N$ being a vector in $\mathbb{R}^N$ whose components
are all $1$.
In \eqref{eq:graph_cond}, the parameters $T > 0$ and $\mu \in (0,1]$ represent
a measure of the level of connectivity of the communication graph.
This condition requires the graph $\Gamma(t)$ to be connected only in an integral
sense, not pointwise in time. Therefore, even if the graph were disconnected during
the mission at some interval of time, the convergence of the "Consensus" algorithm
would still be possible.
With this condition, we can capture also packets dropouts, loss of communication and
switching topologies, which can all occur during the mission, but these events do not necessary
break the convergence property.


\input{chapters/chapter-01/convergence_property.tex}
