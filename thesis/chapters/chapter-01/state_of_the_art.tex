%%Introduction of the chapter
The literature about the "Consensus" is grown significantly in the latest years
because of the increasing presence of autonomous vehicles. This, in accordance with
the new improvements in robotics, has brought a growing interest in consensus
between multiple agents which have to accomplish a mission in cooperative or adversarial
scenarios.

The "Consensus" theory takes its roots in Graph Theory and Automatics and it can
be used to coordinate a mission in order to achieve the synchronization between
the vehicles even when there might be some unforeseens which force one or more
components of the mission to change the planned trajectory or task. In this case,
the other components identify this variation and they will act for preserving the
synchronization.

The "Consensus" algorithm is a distributed algorithm which sometimes can be simulated
in a centralized fashion because of the reduced computational power of the machines
involved in the mission, which are equipped with low power hardware to account for
the crucial issue of power consumption.
In this scenario, a centralized server simulates the algorithm and communicates with
all the machines.

The most common "Consensus" application is a spatial and timing consensus.
Each vehicle of the formation has to travel along a specified trajectory and the
completion of the mission occurs when all the agents reach the final positions of their
spatial paths. The algorithm has to guarantee the difference between
the ending time at which all the vehicles finish their tasks is minimized and
asymptotically goes to zero when the execution time goes to infinity and no other
unforeseen happens.
We also consider formations of Unmanned Aerial Vehicles but the key concepts can
be freely applied to other categories of robots when they are able to follow a trajectory.
This study is focused on this kind of application and further
simulation result and experimental achievements are presented in the chapters
\ref{chap:simulation_results} and \ref{chap:experimental_results}.

In the following we refer to the main work of Venanzio Cichella in this field \cite{cichellaMain}
in order to provide an homogeneous state of the art about the "Consensus".
The paper considers all the details needed to build a "Consensus" system.
We can identify the main components of this kind of systems:
\begin{itemize}
  \item Parametrized trajectory
  \item Virtual time
  \item Consensus law
  \item Network topology
\end{itemize}



%% Sections of the chapter
\section{Parametrized trajectory\label{sec:parametrized_trajectory}}
The trajectory is a spatial path with associated a timing law and it is used
to identify position and orientation of the center of mass of our agent in the space.
We can start with the following definition of a generic trajectory $ p_{d,i}(t_d) $
for $ N $ vehicles:

\begin{equation}  \label{eq:traj_def}
  p_{d,i}:[0,t^f_{d,i}] \rightarrow \mathbb{R}^3, \quad i = 1,2,\dots,N
\end{equation}
where $ t_d \in [0, T_d] $, with $T_d := max \{ t^f_{d,1}, \dots , t^f_{d,N} \} $,
is the time variable of the trajectory, while $ t^f_{d,i} \in \mathbb{R}^+ $
are the individual final mission times of the vehicles obtaining during the planning
phase. Usually all this final times are equal and therefore
$ t^f_{d,1} = \dots = t^f_{d,N} = T_d$, but we introduced the notation
for the sake of completeness.
Obviously the trajectories need to be collision free and must satisfy spatial
and temporal constraints due to the dimension of the vehicle and its maximum
velocities and accelerations.

We now parametrize the trajectory using a dimensionless variable $ \zeta_i \in [0,1]$,
related to the time $t_d$. In this way we can specify a function $ \theta( \cdot )$,
which represents the timing law associated with the spatial path $p_{d,i}(\zeta_i)$.
We can specify this timing law using the dynamic relation of the form:

\begin{equation}  \label{eq:tim_law_def}
  \theta( t_d ) = \frac{d \zeta_i}{dt_d}
\end{equation}

where $ \theta( t_d ) $ is a smooth and positive (the parameter increases when
the time increases) function.

It is desirable that an analytical expression for the function $ \zeta_i (t_d) $
to be available, since it allows a one-to-one correspondence between the time
variable $t_d$ and the parameter $ \zeta_i $.
Using the timing law, defined in \eqref{eq:tim_law_def}, the map $ \zeta_i ( t_d) $
is given by the integral:

\begin{equation}  \label{eq:zeta_law_def}
  \zeta( t_d ) = \int^{t_d}_0 \theta_i(\tau) d \tau
\end{equation}

Usually, all this functions are defined as polynomials in order to make quicker and
easier the evaluation process, since multiplication and addition are the basic
operations in a digital processor, but it is not mandatory to use them and a
generic shape for the functions can be designed.
We have defined all the elements of a trajectory and we do not enter in details
about the trajectory generation phase.
Further details about boundary conditions and flyable trajectory which satisfy
the dynamic constraints of the vehicles can be found in \cite{cichellaMain} and
are extensively presented in \cite{trajGeneration1}, \cite{trajGeneration2},
\cite{trajGeneration3}.


\section{Virtual time\label{sec:virtual_time}}

Given $ N $ collision free trajectories, we want each vehicle to follow a \textit{virtual target},
moving along the path computed offline by the trajectory generation algorithm.
The objective can be achieved introducing a \textit{virtual time}, $ \gamma_i $,
which is used to evaluate the trajectory and can be adjusted online to reach
the synchronization even when external disturbances occur.
Thus, the position of the $ i^{th} $ virtual target is denoted by $ p_{d,i} ( \gamma_i (t))$
and the $ i^{th} $ vehicle tries to follow it, by reducing to zero a suitably defined
error vector using control inputs.

Considering the trajectory $ p_{d,i} (t_d) $ produced by the trajectory generation
algorithm, we consider the virtual time $ \gamma_i $ as a function of time $ t $,
which relates the actual time $t$ to mission planning time $t_d$.

\begin{equation}  \label{eq:virt_time_func}
  \gamma_i : \mathbb{R}^+ \rightarrow [0, T_d], \qquad for \; all \quad i = 1,2,\dots,N
\end{equation}

We can now define the virtual target's position, velocity and acceleration, which
have to be followed by the $ i^{th}$ vehicle at time $t$

\begin{equation}  \label{eq:pos_vel_acc_def}
  \begin{aligned}
  p_{c,i}(t) = \ & p_{d,i}(\gamma_i(t)) \\
  v_{c,i}(t) = \ & \dot{p}_{d,i} (\gamma_i(t), \dot{\gamma}_i (t)) \\
  a_{c,i}(t) = \ & \ddot{p}_{d,i} (\gamma_i(t), \dot{\gamma}_i(t), \ddot{\gamma}_i(t))
  \end{aligned}
\end{equation}

With the above formulation, if $ \dot{\gamma} = 1$, then the speed profile of
the virtual target is equal to the desired speed profile computed at trajectory
generation level.
Indeed, if $ \dot{\gamma} = 1$, for all $t \in [0, T_d]$, with $\gamma_i(0) = 0$,
it implies that $\gamma_i(t) = t_d$ for all $t$ and thus:

\begin{equation*}
  p_{c,i}(t) = p_{d,i}(\gamma_i(t)) = p_{d,i}(t) = p_{d,i} (t_d)
\end{equation*}

In this particular case, the desired and commanded trajectories coincide
in every time instant and also the velocity profiles coincide with the ones
chosen at the trajectory generation time.
If instead $\dot{\gamma_i} > 1$, it implies a faster execution of the mission;
on the other hand, $\dot{\gamma_i} < 1$ implies a slower one.

We can now normalize $\gamma_i$ in order to have a range which is $[0,1]$. We simply
need to divide all by $T_d$. In this way, we could use Bezier curves in order to
represent the spatial path. This kind of curves offer interesting properties for
computing minimum distances between two of them and allow the computation of
smooth trajectories. In any case, it is not mandatory to use them and a general
function could be used instead.

The second order derivative of $\gamma_i$, $\ddot{\gamma}_i$, is a free parameter
used to achieve the consensus. In the next section we will introduce the control law
which commands its evolution during time and we will explain how it is possible to
implement a distributed algorithm.


\section{Consensus law\label{sec:consensus_law}}

Now, we formally state the path following problem. We define as $p_i(t) \in \mathbb{R}^3$
the position of the center of mass of the $i^{th}$ agent and since $p_{c,i}(t)$
describes the commanded position to be followed by the agent at time $t$, the errors
are defined as:

\begin{equation}  \label{eq:pos_vel_acc_err}
  \begin{aligned}
  e_{p,i}(t) = \ & p_{c,i}(t) - p_i(t) \in  \mathbb{R}^3\\
  e_{v,i}(t) = \ & v_{c,i}(t) - \dot{p}_i(t) \in  \mathbb{R}^3
  \end{aligned}
\end{equation}

Then, the objective reduces to that of regulating the error defined in \eqref{eq:pos_vel_acc_err}
to a neighbourhood of zero.
This task is solved with an autopilot capable of following the set points computed
from the desired trajectory at specified instances of time.

The virtual time is the parameter used to reach consensus between multiple vehicles.
In fact, since the the trajectories are parametrized by $\gamma_i$, the agents are
synchronized at time $t$ when:

\begin{equation} \label{eq:diff_gamma}
  \gamma_i(t) - \gamma_j(t) = 0 \quad for \; all \quad i,j \in {1 , \dots , N}, \quad i \neq j
\end{equation}

We can also control the rate of progression of the mission using a parameter
$\dot{\gamma}_d \in \mathbb{R}$, which represents the velocity of the virtual time
with respect to the real time. All the agents share this variable and they proceed
at the same rate of progression if:

\begin{equation} \label{eq:diff_dot_gamma}
  \dot{\gamma}_i(t) - \dot{\gamma}_d(t) = 0 \quad for \; all \quad i \in {1 , \dots , N}
\end{equation}

Adjusting $\dot{\gamma}_d$ we can decide the speed of the mission: for instance
if we set $\dot{\gamma}_d = 1$ and \eqref{eq:diff_gamma} and \eqref{eq:diff_dot_gamma}
are satisfied for all the vehicles, then the mission is executed at the speed
originally planned in the trajectory generation phase.
If instead we use $\dot{\gamma}_d > 1$ or $\dot{\gamma}_d < 1$ we carry out the
mission faster or slower.
This term can be changed in real time in order to get rid of moving objects or
unplanned obstacles which force to change one of the path of the agents.
For the purpose of "Consensus", the parameter is only a reference command,
rather than a control input.

Now we introduce the coordination control law which regulates the evolution of
$\ddot{\gamma}_i(t)$ during the time and determines $\gamma_i(t)$:

\begin{equation} \label{eq:cons_law}
  \begin{aligned}
    \ddot{\gamma}_i(t) = & \; \ddot{\gamma}_d(t) - b (\dot{\gamma}_i(t) - \dot{\gamma}_d(t)) - a \sum_{j \in \aleph_i} (\gamma_i(t) - \gamma_j(t)) - \overline{\alpha}_i (e_{p,i}(t)) \\
    \dot{\gamma}_i(0) = & \; \dot{\gamma}_d(0) = 1 \\
    \gamma_i(0) = & \; \gamma_d(0) = 0
  \end{aligned}
\end{equation}
where $a$ and $b$ are positive coordination control gains, while $\overline{\alpha}_i (e_{p,i}(t)$
is defined as:

\begin{equation} \label{eq:error_term}
  \overline{\alpha}_i (e_{p,i}(t)) = \frac{v_{c,i}^T(t) e_{p,i}(t)}{||v_{c,i}(t)|| + \epsilon}
\end{equation}

with $\epsilon$ being a positive design parameter and $e_{p,i}$ the position
error vector defined in \eqref{eq:pos_vel_acc_err}.
In the equation \eqref{eq:cons_law} we have four terms. The feedforward term
$\ddot{\gamma}_d$ allows the virtual target to follow the acceleration profile of
$\gamma_d$.
The second term $- b (\dot{\gamma}_i(t) - \dot{\gamma}_d(t))$ reduces the error
between the speed profile imposed by $\dot{\gamma}_d(t)$ and $\dot{\gamma}_i(t)$,
which corresponds to the control objective given in \eqref{eq:diff_dot_gamma}.
In particular, if $\dot{\gamma}_d(t)$ is one, then the virtual target converges
to the desired speed profile chosen in the trajectory generation phase.
The third term $- a \sum_{j \in \aleph_i} (\gamma_i(t) - \gamma_j(t))$ ensures that
all the vehicles are coordinated with their neighbors as specified in \eqref{eq:diff_gamma}.
Finally, the fourth term $- \overline{\alpha}_i (e_{p,i}(t))$ is a correction term
used to take into account for the path following errors of the agent. Indeed if the
vehicle is behind its target, the term is not zero and the target slows down in order
to wait the real vehicle.

With this control law we want our vehicles to be synchronized and to proceed at
a desired rate of progression, in order to accomplish the mission even when
some unforeseen disturbances occur during the execution phase.


\section{Network topology\label{sec:network_topology}}


\section{Convergence properties\label{sec:convergence_properties}}

The control law given by \eqref{eq:cons_law} guarantees that the error of the
"Consensus" algorithm converges to zero exponentially.
It can be shown that the maximum convergence rate is given by the sum of
the convergence rate of the path following error and the term

\begin{equation}
  \frac{a}{b} \frac{N \mu}{T ( 1 + (a / b) N T)^2}
\end{equation}

which depends on the control gains $a$ and $b$, the number of vehicles $N$ and
the quality of service of the communication network, characterized by the parameters
$T$ and $\mu$.
If we fix the gains and the number of the vehicles, the convergence rate depends
only on the amount of information which the agents exchange each other over time.

