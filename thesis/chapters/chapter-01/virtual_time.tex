\section{Virtual time\label{sec:virtual_time}}

Given $ N $ collision free trajectories, we want each vehicle follows a \textit{virtual target},
moving along the path computed offline by the trajectory generation algorithm.
The objective can be achieved introducing a \textit{virtual time}, $ \gamma_i $,
which is used to evaluate the trajectory and can be adjusted online to reach
the synchronization even when external disturbances occur.
Thus, the position of the $ i^{th} $ virtual target is denoted by $ p_{d,i} ( \gamma_i (t))$
and the $ i^{th} $ vehicle tries to follow it, by reducing to zero a suitably defined
error vector using control inputs.

Considering the trajectory $ p_{d,i} (t_d) $ produced by the trajectory generation
algorithm, we consider the virtual time $ \gamma_i $ as a function of time $ t $,
which relates the actual time $t$ to mission planning time $t_d$.

\begin{equation}  \label{eq:virt_time_func}
  \gamma_i : \mathbb{R}^3 \rightarrow [0, T_d], \qquad for \; all \quad i = 1,2,\dots,N
\end{equation}

We can now define the virtual target's position, velocity and acceleration, which
have to be followed by the $ i^{th}$ vehicle at time $t$

\begin{equation}  \label{eq:pos_vel_acc_def}
  \begin{aligned}
  p_{c,i}(t) = \ & p_{d,i}(\gamma_i(t)) \\
  v_{c,i}(t) = \ & \dot{p}_{d,i} (\gamma_i(t), \dot{\gamma}_i (t)) \\
  a_{c,i}(t) = \ & \ddot{p}_{d,i} (\gamma_i(t), \dot{\gamma}_i(t), \ddot{\gamma}_i(t))
  \end{aligned}
\end{equation}

With the above formulation. if $ \dot{\gamma} = 1$, then the speed profile of
the virtual target is equal to the desired speed profile computed at trajectory
generation level.
Indeed, if $ \dot{\gamma} = 1$, for all $t \in [0, T_d]$, with $\gamma_i(0) = 0$,
implies that $\gamma_i(t) = t_d$ for all $t$ and thus:

\begin{equation*}
  p_{c,i}(t) = p_{d,i}(\gamma_i(t)) = p_{d,i}(t) = p_{d,i} (t_d)
\end{equation*}

In this particular case, the desired and commanded trajectories coincides
in every time instant and also the velocity profile coincides with the one
chosen at trajectory generation time.
If instead $\dot{\gamma_i} > 1$, it implies a faster execution of the mission;
on the other hand, $\dot{\gamma_i} < 1$ implies a slower one.

The second order derivative of $\gamma_i$, $\ddot{\gamma}_i$, is a free parameter
used to achieve the consensus. In the next section we will introduce the control law
which commands its evolution during time and we will explain how it is possible to
implement a distributed algorithm.
