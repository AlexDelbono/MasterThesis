\section{Network topology\label{sec:network_topology}}

To achieve the time-coordination objective, agents must exchange information
over a supporting communication network.
To analyze the information flow, we need to consider some tools from algebraic graph
theory, whose key concepts can be found in \cite{graphBook}.

We assume that a vehicle $i$ exchanges information with only a subset of all
vehicles, denoted as $\aleph_i(t)$.
We assume that arcs of the network are bidirectional and that there are no
network delays.
The information exchanged is composed by the virtual time of the agents,
$\gamma_i(t)$.

The topology of the graph, $\Gamma(t)$, that represents the communication network,
must comply with the following constraint in order to guarantee the convergence
of the consensus algorithm:

\begin{equation}  \label{eq:graph_cond}
  \frac{1}{NT} \int_t^{t+T} Q L (\tau) Q^T d \tau \ge \mu I_{N-1}, \quad for \; all \quad t \ge 0
\end{equation}
where $L(t) \in \mathbb{R}^{N \times N}$ is the Laplacian of the graph $\Gamma(t)$
and $Q \in \mathbb{R}^{(N-1) \times N}$ is a matrix such that $Q 1_N = 0$ and
$QQ^T = I_{N-1}$, with $1_N$ being a vector in $\mathbb{R}^N$ whose components
are all $1$s.
In \eqref{eq:graph_cond}, the parameters $T > 0$ and $\mu \in (0,1]$ represent
a measure of the level of connectivity of the communication graph.
This condition requires the graph $\Gamma(t)$ to be connected only in an integral
sense, not pointwise in time. Therefore, even if the graph were disconnected during
the mission at some interval of time, the convergence of the consensus algorithm
would still be possible.
With this condition, we can capture also packets dropouts, loss of communication and
switching topologies, which can all occur during the mission, but these events do not necessary
break the convergence property.
