\section{Network topology\label{sec:network_topology}}

To achieve time-coordination objective, the agents must exchange information
over a supporting communication network.
To analyze the information flow we need to consider tools from algebraic graph
theory, of which the key concepts can be found in \cite{graphBook}.

We assume that a vehicle $i$ exchanges information with only a subset of all
vehicles, denoted as $\aleph_i(t)$.
We assume that arcs of the network are bidirectional and that there are no
network delays.
The information exchanged is composed by the virtual time of the agents,
$\gamma_i(t)$.

The topology of the graph $\Gamma(t)$ that represents the communication network
must satisfy the following constrain in order to guarantee the convergence
of the "Consensus" as it will be presented later:

\begin{equation}  \label{eq:graph_cond}
  \frac{1}{NT} \int_t^{t+T} Q L (\tau) Q^T d \tau \ge \mu I_{N-1}, \quad for \; all \quad t \ge 0
\end{equation}
where $L(t) \in \mathbb{R}^{N \times N}$ is the Laplacian of the graph $\Gamma(t)$
and $Q \in \mathbb{R}^{(N-1) \times N}$ is a matrix such that $Q 1_N = 0$ and
$QQ^T = I_{N-1}$, with $1_N$ being a vector in $\mathbb{R}^N$ whose components
are all $1$.
In \eqref{eq:graph_cond}, the parameters $T > 0$ and $\mu \in (0,1]$ represent
a measure of the level of connectivity of the communication graph.
This condition requires the graph $\Gamma(t)$ to be connected only in an integral
sense, not pointwise in time. So, the graph can be disconnected during
the mission in some interval of time, but this could allow the convergence
of the "Consensus" algorithm.
With this condition we can capture also packets dropouts, loss of communication and
switching topologies, which can occur during the mission, but they do not necessary
break the convergence property.
