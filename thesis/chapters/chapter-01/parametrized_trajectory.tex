\section{Parametrized trajectory\label{sec:parametrized_trajectory}}
The trajectory is a spatial path with associated a timing law and it is used
to identify position and orientation of the center of mass of our agent in the space.
We can start with the following definition of a generic trajectory $ p_{d,i}(t_d) $
for $ N $ vehicles:

\begin{equation}  \label{eq:traj_def}
  p_{d,i}:[0,t^f_{d,i}] \rightarrow \mathbb{R}^3, \quad i = 1,2,\dots,N
\end{equation}
where $ t_d \in [0, T_d] $, with $T_d := max \{ t^f_{d,1}, \dots , t^f_{d,N} \} $,
is the time variable of the trajectory, while $ t^f_{d,i} \in \mathbb{R}^+ $
are the individual final mission times of the vehicles obtaining during the planning
phase. Usually all this final times are equal and therefore
$ t^f_{d,1} = \dots = t^f_{d,N} = T_d$, but we introduced the notation
for the sake of completeness.
Obviously the trajectories need to be collision free and must satisfy spatial
and temporal constraints due to the dimension of the vehicle and its maximum
velocities and accelerations.

We now parametrize the trajectory using a dimensionless variable $ \zeta_i \in [0,1]$,
related to the time $t_d$. In this way we can specify a function $ \theta( \cdot )$,
which represents the timing law associated with the spatial path $p_{d,i}(\zeta_i)$.
We can specify this timing law using the dynamic relation of the form:

\begin{equation}  \label{eq:tim_law_def}
  \theta( t_d ) = \frac{d \zeta_i}{dt_d}
\end{equation}

where $ \theta( t_d ) $ is a smooth and positive (the parameter increases when
the time increases) function.

It is desirable that an analytical expression for the function $ \zeta_i (t_d) $
to be available, since it allows a one-to-one correspondence between the time
variable $t_d$ and the parameter $ \zeta_i $.
Using the timing law, defined in \eqref{eq:tim_law_def}, the map $ \zeta_i ( t_d) $
is given by the integral:

\begin{equation}  \label{eq:zeta_law_def}
  \zeta( t_d ) = \int^{t_d}_0 \theta_i(\tau) d \tau
\end{equation}

Usually, all this functions are defined as polynomials in order to make quicker and
easier the evaluation process, since multiplication and addition are the basic
operations in a digital processor, but it is not mandatory to use them and a
generic shape for the functions can be designed.
We have defined all the elements of a trajectory and we do not enter in details
about the trajectory generation phase.
Further details about boundary conditions and flyable trajectory which satisfy
the dynamic constraints of the vehicles can be found in \cite{cichellaMain} and
are extensively presented in \cite{trajGeneration1}, \cite{trajGeneration2},
\cite{trajGeneration3}.
