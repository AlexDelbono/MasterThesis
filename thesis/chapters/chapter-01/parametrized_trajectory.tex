\section{Parametrized trajectory\label{sec:parametrized_trajectory}}
The trajectory is a spatial path with an associated timing law and it is used
to identify the position of the center of mass of our agents at a given time.
First of all, the trajectory used during our experiments are polynomials curves.
In particular, we use Bézier curves, because of the existence of many
computationally efficient algorithms designed for this kind of curves,
such as algorithms to efficiently compute the minimum distance between two Bezier curves
and the existence of a closed-form solution for the arc lengths of the paths.

First of all, let $I = \{N, E, D\}$ denote a right-handed inertial frame with $N$, $E$ and $D$
unit vectors along north, east and down respectively.
The vector $p = (x, y, z) \in I$ denotes the position of the center of mass of the vehicle.
Let $B = \{X_B, Y_B, Z_B\}$ be right-handed body fixed frame centered in the center of mass of the vehicle.
The orientation of the rigid body is given by rotation matrix $R(\phi, \theta, \psi)$ where
$\phi$, $\theta$ and $\psi$ are the roll, pitch and yaw Euler angles respectively.
We can state the following definition of a generic trajectory $ \bm{p}_{d,i}(t_d) $
for $ N $ vehicles:

\begin{equation}  \label{eq:traj_def}
  \bm{p}_{d,i}:[0,t^f_{d,i}] \rightarrow \mathbb{R}^4, \quad i = 1,2,\dots,N
\end{equation}
where $ t_d \in [0, T_d] $, with $T_d := max \{ t^f_{d,1}, \dots , t^f_{d,N} \} $,
is the time variable of the trajectory, while $ t^f_{d,i} \in \mathbb{R}^+ $
are the individual final mission times of the vehicles obtained during the planning
phase. Usually all these final times are equal and therefore
$ t^f_{d,1} = \dots = t^f_{d,N} = T_d$, but we introduced the notation
for the sake of generality.
Obviously, the trajectories need to be collision-free and must comply with spatial
and temporal constraints due to the dimensions of the vehicles and their maximum
velocities and accelerations.
The range of the function is $\mathbb{R}^4$ because, in the case of UAVs, we want
to control $x$, $y$, $z$, $\psi$.
The trajectory can account also for $roll$ and $pitch$, and therefore, the function might take
images in $\mathbb{R}^m$, where $m$ is the number of dimensions considered.
In the following sections, we will refer only to the position and $\psi$, but a more general
theory can also be developed.

We now parametrize the trajectory using a dimensionless variable $ \zeta_i \in [0,1]$,
related to the time $t_d$. In this way we can specify a function $ \theta( \cdot )$,
which represents the timing law associated with the spatial path $\bm{p}_{d,i}(\zeta_i)$.
We can specify this timing law using the dynamic relation:

\begin{equation}  \label{eq:tim_law_def}
  \theta( t_d ) = \frac{d \zeta_i}{dt_d}
\end{equation}

where $ \theta( t_d ) $ is a smooth and positive (the parameter increases when
the time increases) function.

As it allows a one-to-one correspondence between the time
variable $t_d$ and the parameter $ \zeta_i $, an analytical expression
for the function $ \zeta_i (t_d) $ is desirable.
Using the timing law defined in \eqref{eq:tim_law_def}, the map $ \zeta_i ( t_d) $
is given by the integral:

\begin{equation}  \label{eq:zeta_law_def}
  \zeta( t_d ) = \int^{t_d}_0 \theta_i(\tau) d \tau
\end{equation}

Usually, all these functions are defined as polynomials in order to make the evaluation process
quicker and easier, since multiplication and addition are the basic
operations in a digital processor. However, it is not mandatory to use them and a
generic shape for the functions can be designed.

For the spatial path we use quintic Bezier curves for the $x$, $y$ and $z$ coordinates,
while we use a third degree polynomial for the $\psi$ of the drone since we need less
elaborate shapes.
Since we are using drones with a decoupled control of $x$, $y$, $z$ and $\psi$, the
polynomials are completely independent.

The timing law associated to these curves is a Bezier third order polynomial.
So, in our case, we have five polynomials which have to be evaluated in real time
in order to provide the set points. The frequency at which we evaluate the trajectory
is $5 Hz$.

The trajectory which describes the mission of a single drone is given by:

\[
  \begin{cases}
    x_i(\zeta_i) = \sum_{j=0}^{5}{\bar{x}_{i,j} b_j^5(\zeta_i)}\\
    y_i(\zeta_i) = \sum_{j=0}^{5}{\bar{y}_{i,j} b_j^5(\zeta_i)}\\
    z_i(\zeta_i) = \sum_{j=0}^{5}{\bar{z}_{i,j} b_j^5(\zeta_i)}\\
    \psi_i(\zeta_i) = \sum_{j=0}^{3}{\bar{\psi}_{i,j} b_j^3(\zeta_i)}\\
    \zeta_i(t_d) = \sum_{j=0}^{3}{\bar{\zeta}_{i,j} b_j^3(t_d)}\\
  \end{cases}
\]

where $\bar{x}_{i,j}$, $\bar{y}_{i,j}$, $\bar{z}_{i,j}$, $\bar{\psi}_{i,j}$ and $\bar{\zeta}$ $\in \mathbb{R}$
are the desired control points of the spatial path and $b_n^m$ are the (up to degree $m$)
Bernstein basis polynomials.
We do not enter in details about the Bezier curves and Bernstein basis polynomials,
because it is not the objective of this work.

We have defined all the elements of a trajectory and we do not go into detail
about the trajectory generation phase.
Further information about boundary conditions and flyable trajectory, which satisfy
the dynamic constraints of the vehicles, can be found in \cite{cichellaMain} and
is extensively analyzed in \cite{trajGeneration1}, \cite{trajGeneration2},
\cite{trajGeneration3}.
