\section{Consensus law\label{sec:consensus_law}}

Now, we formally state the path following problem. We define as $\bm{p}_i(t) \in \mathbb{R}^4$
the position of the center of mass of the $i^{th}$ agent and its $yaw$ and since $\bm{p}_{c,i}(t)$
describes the commanded pose to be followed by the agent at time $t$, the errors
are defined as:

\begin{equation}  \label{eq:pos_vel_acc_err}
  \begin{aligned}
  \bm{e}_{p,i}(t) = \ & \bm{p}_{c,i}(t) - \bm{p}_i(t) \in  \mathbb{R}^4\\
  \bm{e}_{v,i}(t) = \ & \bm{v}_{c,i}(t) - \dot{\bm{p}}_i(t) \in  \mathbb{R}^4
  \end{aligned}
\end{equation}

Then, the objective reduces to that of regulating the error defined in \eqref{eq:pos_vel_acc_err}
to a neighbourhood of zero.
This task is solved with an autopilot capable of following the set points computed
from the desired trajectory at specified instances of time.

The virtual time is the parameter used to reach consensus between multiple vehicles.
In fact, since the the trajectories are parametrized by $\gamma_i$, the agents are
synchronized at time $t$ when:

\begin{equation} \label{eq:diff_gamma}
  \gamma_i(t) - \gamma_j(t) = 0 \quad for \; all \quad i,j \in {1 , \dots , N}, \quad i \neq j
\end{equation}

We can also control the rate of progression of the mission using a parameter
$\dot{\gamma}_d \in \mathbb{R}$, which represents the velocity of the virtual time
with respect to the real time. All the agents share this variable and they proceed
at the same rate of progression if:

\begin{equation} \label{eq:diff_dot_gamma}
  \dot{\gamma}_i(t) - \dot{\gamma}_d(t) = 0 \quad for \; all \quad i \in {1 , \dots , N}
\end{equation}

Adjusting $\dot{\gamma}_d$, we can decide the speed of the mission: for instance,
if we set $\dot{\gamma}_d = 1$ and \eqref{eq:diff_gamma} and \eqref{eq:diff_dot_gamma}
are satisfied for all the vehicles, then the mission is executed at the speed
originally planned in the trajectory generation phase.
If instead we use $\dot{\gamma}_d > 1$ or $\dot{\gamma}_d < 1$ we carry out the
mission faster or slower.
This term can be changed in real time in order to avoid moving objects or
unplanned obstacles, which make it necessary to change one of the path of the agents.
For the purpose of “Consensus”, the parameter is only a reference command,
rather than a control input.

Now, we introduce the coordination control law which regulates the evolution of
$\ddot{\gamma}_i(t)$ during the time and determines $\gamma_i(t)$:

\begin{equation} \label{eq:cons_law}
  \begin{aligned}
    \ddot{\gamma}_i(t) = & \; \ddot{\gamma}_d(t) - b (\dot{\gamma}_i(t) - \dot{\gamma}_d(t)) - a \sum_{j \in \aleph_i} (\gamma_i(t) - \gamma_j(t)) - \overline{\alpha}_i (\bm{e}_{p,i}(t)) \\
    \dot{\gamma}_i(0) = & \; \dot{\gamma}_d(0) = 1 \\
    \gamma_i(0) = & \; \gamma_d(0) = 0
  \end{aligned}
\end{equation}
where $a$ and $b$ are positive coordination control gains, while $\overline{\alpha}_i (\bm{e}_{p,i}(t))$
is defined as:

\begin{equation} \label{eq:error_term}
  \overline{\alpha}_i (\bm{e}_{p,i}(t)) = \frac{\bm{v}_{c,i}^T(t) \bm{e}_{p,i}(t)}{||\bm{v}_{c,i}(t)|| + \epsilon}
\end{equation}

with $\epsilon$ being a positive design parameter, $\bm{e}_{p,i}$ the position
error vector defined in \eqref{eq:pos_vel_acc_err} and $\aleph_i$ the set of the
neighbors which can communicate with the $i^{th}$ vehicle (we will see details later).
In the equation \eqref{eq:cons_law} we have four terms. The feedforward term,
$\ddot{\gamma}_d$, allows the virtual target to follow the acceleration profile of
$\gamma_d$.
The second term, $- b (\dot{\gamma}_i(t) - \dot{\gamma}_d(t))$, reduces the error
between the speed profile imposed by $\dot{\gamma}_d(t)$ and $\dot{\gamma}_i(t)$,
which corresponds to the control objective given in \eqref{eq:diff_dot_gamma}.
In particular, if $\dot{\gamma}_d(t)$ is one, then the virtual target converges
to the desired speed profile chosen in the trajectory generation phase.
The third term, $- a \sum_{j \in \aleph_i} (\gamma_i(t) - \gamma_j(t))$, ensures that
all the vehicles are coordinated with their neighbors as specified in \eqref{eq:diff_gamma}.
Finally, the fourth term, $- \overline{\alpha}_i (\bm{e}_{p,i}(t))$, is a correction term
used to take into account for the path following errors of the agent. Indeed, if the
vehicle is behind its target, the term is not zero and the target slows down in order
to wait for the real vehicle.

With this control law, we want our vehicles to be synchronized and to proceed at
a desired rate of progression, in order to accomplish the mission even when
some unforeseen disturbances occur during the execution phase.
