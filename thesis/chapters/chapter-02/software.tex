\section{Software}

We now list the software adopted to execute the algorithm
in a real environment and in the simulated one.

The principal software used to manage the distributed architecture is ROS Kinetic
Kame \cite{ros}.
ROS is a robotic middleware with a structure which is mainly publisher-subscriber
which can manage more machines in a distributed environment.
The central part of the ROS architecture is a node called ROS core, which manages
the topics of the system and the subscriptions.
The ROS core offers also other functionalities, such as the Parameter Server or the
possibility to advertise services.
The Parameter Server is a central infrastructure, which is responsible for storing
configuration parameters loaded by the nodes of the system.
These parameters can be retrieved by other nodes and used if necessary.
Instead, a ROS service is a sort of remote function call. One node can advertise
the service, which can be called by any other node. The call is synchronous,
so the caller is blocked until the callee has executed its callback function.
The ROS architecture is based on queues, threads and callback functions, but
most of the provided tools hide part of the implementations of the distributed
environment.


\subsection{Ground station}
The ground station runs Windows 10 Pro \cite{windows} and the software used to virtualize a
Desktop machine is VMware \cite{vmware}.
On the virtual machine is installed Ubuntu 16.04 LTS \cite{ubuntu} in order to run software
needed and available only for Unix systems.

On Windows operating system we launch the Motive Optitrack software \cite{optitrack},
which allows to calibrate and control the cameras for position tracking.
It then provides the streaming of the positions of the markers identified
by the cameras and sends it to the Ubuntu operating system using a multicast IP address.
Here, the information is converted by a ROS node and sent through the ROS topics,
which are read by the drones. In this way, each drone knows exactly its position.
This conversion node is an open source node called Mocap which can be found on GitHub \cite{mocap}.
On Ubuntu side, we launch the ROS core, which manages all the ROS nodes and topics.

\subsection{Raspberry Pi Zero}
The Raspberry Pi Zero executes a dedicated version of Debian operating system,
which is Raspbian. The version used is Raspbian Jessie 4.4 \cite{raspbian}.

\subsection{Intel Edison}
The Intel Edison runs a version of Debian called Jubilinux, at version 0.1.1 \cite{jubilinux}.

\subsection{Both companions}
Both companions, the Raspberry and the Edison, are provided with ROS Kinetic and
both have to execute some ROS nodes in order to communicate with the other drones.

Both of them run Mavros nodes \cite{mavros}.
which can be downloaded from GitHub and manage the conversion of the
information taken from the ROS topics to the serial port and vice versa. Indeed,
the ROS messages are converted into Mavlink messages and sent through the serial port
to the Pixfalcon autopilot. The same is done for the Mavlink messages from the
autopilot, which are published on ROS topics.

The second kind of ROS node run by the companions is a custom consensus node, which
loads the desired trajectory and sends the next set point to the Mavros node.
This node will be analyzed in detail in Chapter \ref{chap:consensus_node}.

\subsection{Pixfalcon}
The Pixfalcon FCU is flashed with PX4 Pro Autopilot \cite{px4},
an open source firmware downloadable from GitHub.
The release used is the v1.5.5.

\subsection{Simulation}
The simulation part is developed with the utilities provided by the PX4 firmware.
In particular, the physical engine used is Gazebo \cite{gazebo}, which is an Open
Source project.

\subsection{Additional software}
We use Matlab R2016\_B \cite{matlab} to process the data, to plot the graphs and
to validate some theoretical results.

This document is written in  \LaTeX \ \cite{latex} and the code IDE used is Atom \cite{atom},
while the versioning control platform used are GitHub \cite{github} and GitLab \cite{gitlab}.
