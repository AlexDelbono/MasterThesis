In this chapter we will address the deployment of the previous state of the art
for our specific case. In particular, we will use a formation of UAVs, which is
composed by heterogeneous vehicles with autopilot.

First of all, the trajectory used during our experiments are polynomials curves.
In particular, we use Bézier curves, because of the existence of many
computationally efficient algorithms designed for this kind curves,
such as algorithms to efficiently compute the minimum distance between two Bezier curves
and the existence of a closed-form solution for the arc lengths of the paths.
As we have specified in the section \ref{sec:parametrized_trajectory}, we decouple the
trajectory into a spatial path, $p_{d,i}(\zeta_i)$, and an associated timing law.

For the spatial path we use quintic Bezier curves for the $x$, $y$ and $z$ coordinates,
while we use a third degree polynomial for the $yaw$ of the drone since we need less
elaborated trajectories.
Since we are using drones with a decoupled control of $x$, $y$, $z$ and $yaw$, the
polynomials are completely independent.
The timing law associated to these curves is a Bezier third order polynomial.
So, in our case, we have five polynomials which have to be evaluated in real time
in order to provide the set points. The rate at which we evaluate the trajectory
is $5 Hz$.

The trajectory which describes the mission of a single drone is shown in the
following system.

\[
  \begin{cases}
    x_i(\zeta_i) = \sum_{j=0}^{5}{\bar{x}_{i,j} b_j^5(\zeta_i)}\\
    y_i(\zeta_i) = \sum_{j=0}^{5}{\bar{y}_{i,j} b_j^5(\zeta_i)}\\
    z_i(\zeta_i) = \sum_{j=0}^{5}{\bar{z}_{i,j} b_j^5(\zeta_i)}\\
    yaw_i(\zeta_i) = \sum_{j=0}^{3}{\bar{yaw}_{i,j} b_j^3(\zeta_i)}\\
    \zeta_i(t_d) = \sum_{j=0}^{3}{\bar{\zeta}_{i,j} b_j^3(t_d)}\\
  \end{cases}
\]

$\bar{x}_{i,j}$, $\bar{y}_{i,j}$, $\bar{z}_{i,j}$, $\bar{yaw}_{i,j}$ and $\bar{\zeta}$ $\in \mathbb{R}$
are the desired control points of the spatial path and $b_n^m$ are the (up to degree $m$)
Bernstein basis polynomials.
We do not enter in details about the Bezier curves and Bernstein basis polynomials,
because it is not the objective of this work.
